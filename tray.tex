\chapter{Abstract}

\chapter{Introduction} \label{ch:introduction}
\section{Background of the Study} % purpose, significance, research question??

\begin{quote}
    "Many forms of Government have been tried, and will be tried in this world of sin and woe. No one pretends that democracy is perfect or all-wise. Indeed, it has been said that democracy is the worst form of Government except all those other forms that have been tried from time to time [...]"
    
    Sir Winston Churchill (\cite{HouseofCommons.1947})
\end{quote}

For millennia, the vast majority of the world had been ruled my autocrats and aristocrats -- by authorities lacking any democratic legitimacy for their power. Rulers appeared to have ultimate power and no accountability, arguably leading to policies and approaches of ruling which, first and foremost, benefit themselves or their social class.

The term "democracy" derives from the two Greek words \textit{demos} (the people) and \textit{kratos} (power) and has therefore a root meaning of "the power of the people" (\cite{Ober.2007}. "Democracy is founded on the right of everyone to take part in the management of public affairs [...]" (\cite{InterParliamentaryUnion.1997}, number 11), in practice mostly in the form of elected officials. These officials are typically either legislators (e.g. \textit{Representatives} and \textit{Senators} in the \acrshort{us} or \textit{Members of the European Parliament} in the \acrshort{eu}) or political executives (e.g. presidents, governors, or prime ministers).

The Universal Declaration on Democracy states that "[p]ublic accountability, which is essential to democracy, applies to all those who hold public authority, whether elected or non-elected, and to all bodies of public authority without exception [...]" (\cite{InterParliamentaryUnion.1997}, number 14), primarily by holding elections (cf. \cite{InterParliamentaryUnion.1997}, number 12).

% Why Is Democratic Innovation Necessary? https://institute.global/policy/what-state-debate-digital-democracy







\chapter{Digital Democracy}
\section{The use of digital democracy around the world}

\subsection{Voting machines in the \acrshort{us}}
The \acrfull{us} maintains an electoral system with many distinct characteristics that differ from most other modern democracies. Citizens in most countries typically only have one or two votes on a ballot, for example only for the president in French presidential elections, or two votes in the \acrfull{mmpr} system in German parliamentary elections. However, the \acrshort{us} has the longest ballots in the democratic world due to federalism in combination a large number of directly publicly offices (\cite{MITElectionLab+ScienceLab.}). A sample ballot from the general election on November 3, 2020 as used Weld County, Colorado can be found in the appendix. This exemplary ballot included votes for the presidential ticket, one \acrshort{us} senator, the \acrshort{us} congressperson for the constituency, state legislators, a county commissioner and council, town councils, the mayor, several judicial retention questions, a total of 11 ballot questions for state measures, as well as a number of municipal and school district measures (\cite{Williams.2020}).

To deal with with this amount of votes that need to be tallied in each election, voting machines have been crucial in U.S. elections for over a century. Mechanical solutions include lever machines and punch-card voting devices, both of which are no longer in use in federal elections. Current forms include scanned paper ballots and -- nowadays most importantly -- \acrfull{dre} voting machines. The latter are digital voting computers on which voters indicate their vote by using either a touchscreen or buttons. While some systems only record votes on internal memory units, \acrshort{dre}s increasingly include a so-called \acrfull{vvpat}. This means that voters can see printed, physical proof of their vote that they can use to verify their casted vote. This paper-trail can also be used as a physical backup for election audits and recounts.

After the 2020 presidential election in which former president Donald Trump has increasingly raised concerns about fraud and malfunctioning voting machines in the U.S. and fueled a widespread suspicion towards voting machines among his supporters (\cite{Giles.2020}). Calls for \acrshort{vvpat} mandates for voting machines in the \acrshort{us} have become more popular following this incident. Previously in the 2000 presidential election, malfunctioning punch-card voting machines have caused chaos in the state of Florida, which led to George W. Bush winning the presidential race against Al Gore with a very small margin (\cite{Kettle.2001}). Even though another technology was involved, both events have caused distrust of automatic voting devices in the \acrshort{us}.


% CHECK ob das am Anfang noch rein sollte und ob das so eingehalten wurde: Foley 2006, p. 174: Participation is a necessary but insufficient condition for workplace democracy (Adams and Hansen, 1992; Cheney, 1995; Collins, 1997; Knudsen, 1995). Workplace democracy exists when employees have some real control over organizational goal-setting and strategic planning, and can thus ensure that their own goals and objectives, rather than only those of the organization, can be met. Participation does not meet the requirements for workplace democracy, because it exists whenever employees are allowed to give input into organizational decisions, even if it means they only suggest ways to implement decisions that have already been made.